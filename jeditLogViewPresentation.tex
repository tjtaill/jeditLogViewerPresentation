\documentclass{beamer}

\usepackage{listings}

\usepackage{enumitem}
\setlist[enumerate]{label*=\arabic*.}

\title{Jedit Log Viewer}
\author{Troy Taillefer}
\institute{Broadsoft}

\date{\today}

\begin{document}
\setbeamertemplate{bibliography item}[text]
\setbeamertemplate{frametitle continuation}{}
\lstset{
basicstyle=\tiny,
language=Java,
showstringspaces=false,
keywordstyle=\bfseries\color{green!40!black},
commentstyle=\itshape\color{purple!40!black},
stringstyle=\color{orange},
}

\frame{\titlepage}

%Forecast
% high leve description of what problem was attacked and major insight
\begin{frame}[fragile,allowframebreaks]{Introduction}
The goal of the project was to provide an excellent tool for analyzing xs logs 
for both general understanding and trouble shooting purposes. To meet this goal three
principles features were developed

\begin{enumerate}
\item Search
\item Level of Detail
\item Navigation/Discovery
\end{enumerate}
\end{frame}

% Outline not sure what goes here Toc
\begin{frame}[fragile,allowframebreaks]{Outline}
\begin{enumerate}
\item Background
\begin{enumerate}
\item Motivation
\item Related Work
\item Search
\item Level of Detail
\item Navigation/Discovery
\end{enumerate}
\item Results
\item Summary
\item Future Work
\end{enumerate}
\end{frame}


%Background
% Motivation and Problem statement (1-2) Slides
\begin{frame}[fragile,allowframebreaks]{Motivation}
XS Logs are not easy to understand and 
it is painful to stare at mono color text. I wanted to have some of the same facilities
I have when looking at code syntax coloring, folding, navigation and search. 
\end{frame}

% Related Work 1 slide
\begin{frame}[fragile,allowframebreaks]{Related Work}

\begin{itemize}
\item LogViewer
\item Loogle\cite{Loogle}
\item SipSpider
\end{itemize}

\end{frame}


% Methods (1-3) Slides
% Explain approach (How)
\begin{frame}[fragile,allowframebreaks]{Search}
Search is the quickest way to get what you want far more efficient then Navigation. Search offers very little in the way of discoverability and is very unfriendly to novices. This project implements search using the Jedit\cite{Jedit} standard text editor search facility as well use Jedit's Beanshell\cite{Beanshell} macro facility to store and hot key pre canned useful searches. 
\end{frame}

\begin{frame}[fragile,allowframebreaks]{Example Beanshell Search Macro}
\begin{lstlisting}
SearchAndReplace.setSearchString("^INVITE sip:");
SearchAndReplace.setAutoWrapAround(true);
SearchAndReplace.setRegexp(true);
SearchAndReplace.setSearchFileSet(new CurrentBufferSet());
SearchAndReplace.find(view);
\end{lstlisting}
\end{frame}

\begin{frame}[fragile,allowframebreaks]{Level of Detail}
XS logs contain a lot of information, the relavance of this information is varies quite a bit therefore it is useful to easily hide irrelvant information. The three principle ways this is achieved in this project are 

\begin{enumerate}
\item Folding via a Jedit Plugin I wrote called XsLogFolder\cite{XSLogFolder} 
\item Filtering the XS logs removing unwanted information through Beanshell Macros
\end{enumerate}

XsLogFolder\cite{XsLogFolder}
\end{frame}

% Results (2-6) Slides
% Present Key results this is main body of talk don't over do it

% Summary 1 slide

% Future Work 1 slide

% Backup Slides (0-3) Slides
% Have some slides to prepared to answer questions or in case you are under time


\begin{frame}[allowframebreaks]{References}
\bibliographystyle{plain}
\bibliography{references}
\end{frame}

\end{document}